\cvsection{Work Experience}

\begin{cventries}

  \cventry
    {Soderling Laboratory, Duke University} % mono-upper
    {Graduate Research Assistant} % bold-left
    {Fall 2016 - April 2021} % italics-right
    {Department of Neurobiology} % Blue-right
    {
      \begin{cvitems} % Description(s) of tasks/responsibilities
      \item {Development of spatial proteomics analysis pipeline for clustering and linear model-based 
	      inference in multiplex spatial proteomics.}
      \item {Recipient of the Ruth L. Kirschstein National Research Service Award (NRSA)
	      Individual Predoctoral Fellowship (F31) to study the molecular
	      mechanisms of a rare epilepsy disorder.
	      }
      \item {Application and development of CRISPR-based methods for genetic
	      depletion and tagging endogenous proteins.}
      \item {Performed synapse biofractionation and proteomics to assess synaptic mechanisms underlying
	      mouse models of genetic neurodevelopmental disorders.}
      \item {Appliction and analysis of BioID proximity proteomics for identification of subcellular
	      specific proteomes such as the postsynaptic proteomes of
	      excitatory and inhibitory synapses.
	      }
      \end{cvitems}
    }

  \cventry
    {Soderling Laboratory, Duke University}
    {Research Technician}
    {May 2014 - May 2016} 
    {Department of Cell Biology} 
    {
      \begin{cvitems} % Description(s) of tasks/responsibilities
      \item {Protein co-immunoprecipitation and proximity proteomics for
	      identification of protein-protein interactions and assessing a protein's interactome.}
      \item {Established use of CRISPR-based tools for gene depletion and
	      interrogation of synaptic protein function. }
      \item {Performed immunoblotting, immunostaining, and cell and tissue
	      culture for assessing protein expression and localization
	      \textit{in vitro} and \textit{in vivo}.}
      \item {Molecular cloning and generation of adeno associated virus for transgene expression
	      \textit{in vivo}.} 
      \item {Maintained mouse colony with >30 strains of mice; performed mouse
	      surgery, husbandry, genotyping, and analysis of mouse behavior.}
      \end{cvitems}
    }

\end{cventries}
